\documentclass[a4paper]{article}
\usepackage{fancyvrb}
\title{Example: Firm-Bank relationship in bad conditions}
\begin{document}
\maketitle

\subsubsection{Bad condition}
\subsubsection*{Step 1: interests and ask for repayments}
Firms can be customers of several banks.

First of all, the banks account the interest rate.

Suppose that after accounting interests we have the following situation
\begin{verbatim}
bank 1 account =   10
bank 2 account = -150
bank 3 account =  -50
\end{verbatim}

The bank assumes that indebted firms ask for the whole renewal of the debt:

\begin{verbatim}
bank 1 account =   10  demanded credit =    0
bank 2 account = -150  demanded credit = -150
bank 3 account =  -50  demanded credit =  -50
\end{verbatim}

Each bank with a negative account can ask for refunding. In this case the allowed credit is lower (in absolute value) to the demanded credit.
Suppose bank 2 asks for refunding and bank 3 does not:

\begin{verbatim}
bank 1 account =   10  demanded credit =    0 allowed credit =    0
bank 2 account = -150  demanded credit = -150 allowed credit = -130 
bank 3 account =  -50  demanded credit =  -50 allowed credit =  -50
\end{verbatim}

In this example, the firm needs 20 to satisfy banks requests.

\subsubsection*{Step 2: refunding}

The possibility to refund depends on the resources available on banks and on the economic result. In our example, 10 is available in bank 1.

To go on with our example, let us assume that the economic result is \verb+-50+ i.e. the firm is suffering a loss.

The firm use 10 available in bank 1, but it is not enough to satisfy banks requests. Shortages are recorded as unpaid amounts.

The firm financial situation is represented as follows

\begin{verbatim}
bank 1: account =    0  demanded credit =    0 allowed credit =    0 unpaid =  0
bank 2: account = -140  demanded credit = -150 allowed credit = -130 unpaid = 10
bank 3: account =  -50  demanded credit =  -50 allowed credit =  -50 unpaid =  0

cashOnHand = -50
\end{verbatim}

\subsubsection*{Step 3: account resetting}
In this step, banks set the demanded and allowed credit are set to zero.

The new situation is

\begin{verbatim}
bank 1: account =    0  demanded credit =  0 allowed credit = 0 unpaid =  0
bank 2: account = -140  demanded credit =  0 allowed credit = 0 unpaid = 10
bank 3: account =  -50  demanded credit =  0 allowed credit = 0 unpaid =  0

cashOnHand = -50
\end{verbatim}



\subsubsection*{Step 4: set desired credit}

Now the firm can ask for new credit. This can be done for two reasons: 1) to finance new investments and 2) to pay unsatisfied lenders.

Suppose now that our firm does not invest, and asks for credit to pay unsatisfied lenders.

Credit in this step is asked to one of the banks, in particular to that with the ``best'' account.

In this example, the new asked credit is \verb/10+50=60/. The update situation is 

\begin{verbatim}
bank 1: account =    0  demanded credit = -60 allowed credit = 0 unpaid =  0
bank 2: account = -140  demanded credit =   0 allowed credit = 0 unpaid = 10
bank 3: account =  -50  demanded credit =   0 allowed credit = 0 unpaid =  0

cashOnHand = -50
\end{verbatim}






\subsubsection*{Step 5: credit supply}

The bank now decides the allowed credit.

The situation evolves differently according to the allowed amount.

Suppose first, the bank allows all the demanded credit. The situation is as follows

\begin{verbatim}
bank 1: account =    0  demanded credit = -60 allowed credit = -60 unpaid =  0
bank 2: account = -140  demanded credit =   0 allowed credit =   0 unpaid = 10
bank 3: account =  -50  demanded credit =   0 allowed credit =   0 unpaid =  0

cashOnHand = -50
\end{verbatim}

\subsubsection*{Step 6: the firm adjust bank accounts}

The resources made available by bank 1 are used and the situation evolve into the following

\begin{verbatim}
bank 1: account =  -60  demanded credit = -60 allowed credit = -60 unpaid =  0
bank 2: account = -130  demanded credit =   0 allowed credit =   0 unpaid =  0
bank 3: account =  -50  demanded credit =   0 allowed credit =   0 unpaid =  0

cashOnHand = 0
\end{verbatim}

\end{document}
